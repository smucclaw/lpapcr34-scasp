\documentclass{IOS-Book-Article}

\usepackage{listings}
\lstset{
basicstyle=\small\ttfamily,
columns=flexible,
breaklines=true,
}
\usepackage{soul}\setuldepth{article}
\usepackage{hyperref}
\usepackage{natbib}
\usepackage{mathptmx}

\setcitestyle{square}
\def\hb{\hbox to 10.7 cm{}}

%\usepackage{times}
%\normalfont
%\usepackage[T1]{fontenc}
%\usepackage[mtplusscr,mtbold]{mathtime}
%
% \usepackage[utf8]{inputenc}
% \usepackage{longtable}
% \usepackage{blindtext,alltt}

\begin{document}

\pagestyle{headings}
\def\thepage{}

\begin{frontmatter}              % The preamble begins here.

%\pretitle{Pretitle}

%% Full title
\title{L4: PL Theory Meets Computational Law}

% order from JURIX paper -- to change later
% \author{{\snm{Wong} \fnms{Meng Weng}}%
% \thanks{Corresponding Author: WONG Meng Weng, Singapore Management University, School of Law, Centre for Computational Law; E-mail: mwwong@smu.edu.sg.}},
% \author{\fnms{Jerrold} \snm{Soh}},
% \author{\snm{Lim} \fnms{How Khang}},
% \author{\fnms{Inari} \snm{Listenmaa}},
% and
% \author{\fnms{Martin} \snm{Strecker}}

\runningauthor{WONG M.W.}
\runningtitle{PL Theory Meets Computational Law}
\address{Singapore Management University}

\begin{abstract}
We demonstrate an early release of ``L4'', a working, practitioner-oriented, open-source domain-specific language for drafting laws and contracts. Viewing a contract as a specification for a multi-agent, concurrent, rule-based distributed system, L4 builds on prior formalisms including LTL/CTL temporal logics, deontic and epistemic modal logics, and process calculi. Using a toy example, we introduce L4's syntax and describe natural language generation, transpilation, formal verification, and metadata embedding into PDF.
\end{abstract}

\begin{keyword}
computational law\sep modal logics\sep domain-specific languages
\end{keyword}
\end{frontmatter}
\markboth{}{September 2020\hb}
\maketitle

\section{Notes for writers}

\textit{This section will not appear in the final publication.}

\subsection{Target Audience}

Who will read this?

law.mit.edu readership; people active in the #RulesAsCode movement; industry practitioners at LexisNexis, ThomsonReuters etc; progressive thought leaders at law firms eg. Mishcon; thought leaders in legaltech companies.

Who do we want to read this?

Goal: We want to recruit masters and PhD students in CS to come work on this project with us.

Goal: We want to increase visibility and mindshare among potential early adopters of our technology.

\subsection{Target Publication}

This is meant for the MIT Computational Law Report. Articles should be upwards of 7,500 words. They don't use a house citation style, so we can stick with computer science (endnote) rather than law (footnote) citations. We should generally write with \textit{both} computer scientists and lawyers in mind, though the former is the \textit{primary} audience, and any lawyer who reads this should be slightly more tech-oriented anyway.

\section{Introduction}

In the past forty years, numerous legal formalisms have emerged from academia; none are widely used, not the way photographers use Photoshop, architects use AutoCAD, and (just about) everybody uses spreadsheets. While spreadsheets ``amplify human capabilities''\footnote{Douglas Engelbart, 1958} in quantitative reasoning, nothing comes close in the adjacent domain of legal and qualitative reasoning. Accountants use Excel very differently from the way lawyers use Word.

[broad idea on structure: first half is a position paper; second half is a demo paper sketching out current work on a solution to some of the issues identified in first half; third half is a list of open problems and challenges.]

[caveats] We exclude the schools of legal informatics which apply machine learning and text analytics methods to jurimetrics and contract analytics.

\section{Case Study}

\textit{In this section, we give a motivating case study, show the L4 code, and show the various targets: NLG, FV errors and warnings, visualization, and automated production of an expert system interview via a modern web UI. Then we promise to explain the underlying logic, semantics, and syntax. Note that further down in this document there may be suitable material.}

\subsection{The Original: a Simple Contract With Rule-Like Elements}

\textit{In this section, we devise a motivating case which is strongly relatable for most people, but also just-so-happens to exercise all the interesting aspects of our language. We show what the legal-drafter version looks like, written in classic legalese.}

\subsection{As Code}
\textit{In this section, we show the L4 restatement without detailed explanation.}

\subsection{As Visualization}
\textit{In this section, we show a visualization of the state graph and of the constitutive elements, emphasizing that it was automatically generated.}

\subsection{As Web UI Expert System}
\textit{In this section, we show a screenshot of the UI in-progress, and a result page answering a user question, with explanation.}

\subsection{Inside an IDE}
\textit{In this section, we show screenshots of an IDE (VS Code) demoing syntax highlight and live output of NLG and visualization.)}

\subsection{Natural Language Generation}
\textit{In this section, we show NLG output in at least two languages, of the L4 code.}

\subsection{Formal Verification Errors and Warnings}
\textit{In this section, we show warnings and errors generated by the more formal components of our compiler – syntax errors, semantic errors, pragmatic errors.}

\subsection{Test Suite}
\textit{In this section, we show how "unit" tests are written in L4, and how a run of the test harness shows: conflicts and omissions in DMN; and violations of LTL/CTL property assertions.}

\section{Background}

The history of computational law is coextensive with the history of artificial intelligence, with roots going back to Leibniz's notion of a \textit{calculus ratiocinator} in the late 17th century. In the modern era, legal theorists have sought to place law on a more formal footing: perhaps inspired by Whitehead \& Russell (1910), Hohfeld (1917) borrowed from Aristotelian logic; Allen (1952, 1978) borrowed from propositional logic; and Darmstadter (2010) sought relief in the practices of software development. 

Coming from the opposite direction, mathematicians and computer scientists have used defeasible logics, default logics, modal logics, and process calculi to characterize problems in legal formalisation and reasoning. (Governatori, Sartor, Prakken). At Imperial, Kowalski and Sadri continue to champion Prolog for law. (British Nationality Act as a Logic Program 1985, Logic Production Systems 2019.) Others have focused their energies on structuring and standardizing normative text in the form of RulemL, LegalRuleML, LegalDocumentML, ContractLog, eContracts XML; yet others exploit advances in natural language processing to extract meaning from normative text. (Estrella, MIREL) Standards such as SBVR and BPMN offer frameworks for expressing business rules, which are close cousins to legal rules. 

Even more recently, the ``Rules as Code'' movement provides a handful of case studies in language design: from the Dutch Tax Code project involving JetBrains MPS, to Python-based OpenFisca used in France and New Zealand among others, to the newly designed Catala language.

In this section, we trace the development of computational law both as an academic field as well as a practical enterprise. Section \ref{sec:what_is} clarifies definitions. Section \ref{}
    
\subsection{What is Computational Law}
\label{sec:what_is}

Our definition of CompLaw follows the vision set by Michael Genesereth but also the past and present of conferences like ICAIL, Jurix, and ReMeP. [move into history of Computational Law]

\textit{In this section, we give our house definition of CompLaw, saying we focus on the symbolic, GOFAI bits; and we clarify that we don't use ``computational law'' in the same way as cohubicol, who focus only on ML/blockchain: https://www.cohubicol.com/about]}

\subsection{Context in Computer Science, what concepts are relevant, etc}

\begin{itemize}
    \item deontic logics and automata
    \item temporal logics and LTL/CTL assertions
    \item specification languages and tools like Alloy, TLA+, UPPAAL, NuSMV, Z3
    \item constraint satisfaction problems and planning problems
    \item constitutive rules, the lambda calculus, and logic programming
    \item default rules and defeasible logic: reductions to first order logic
    \item macros, metaprogramming, and homoiconicity
    \item computational semantics, questions of literal vs purposive interpretation of legal text
    \item controlled natural language and multilingual isomorphism
\end{itemize}

\subsection{Recent and existing work}

CoSoDis. MIREL Project. The work of Gerardo Schneider and Aarne Ranta. Work at Chalmers. Data61.

Distinction between specification languages and implementation languages; extraction as the bridge between the two.

\subsubsection{Symboleo}

\subsubsection{Deon Digital's CSL}

as an example of a blockchain-oriented implementation language; possibly others from Meng's chapter on Smart Legal Contracts vs Legal Smart Contracts; mention Accord Project, Hyperledger, Cicero, Ergo; mention Tezos and Michelson.

\subsubsection{FCL}

\subsubsection{Catala and OpenFisca}

\subsubsection{LegalRuleML and LegalDocumentML}

\subsubsection{FormaLex}

\subsubsection{Attempto and ACERules}

\subsubsection{Flora-2}

\subsubsection{LPS}

\subsubsection{BlawX}

\subsubsection{Drools, DMN, BPMN}

\subsection{Industry Adoption}

Every chapter from \textit{Computer Science and Law} has matured into its own industry.

\subsubsection{Industry implementations}

The Rules As Code movement; what tools specifically are they using?

Remark from Hillel Wayne's findings -- two unique aspects of software engineering: community and version control; how these relate to computational law and LegalTech innovation.

\subsubsection{What lawyers have said}
\textit{jerrold working on this}

markou and computability:

https://papers.ssrn.com/sol3/papers.cfm?abstract_id=3589184 and the book

https://www.bloomsburyprofessional.com/uk/is-law-computable-9781509937066/

megan ma and representation: 

https://law.mit.edu/pub/writinginsign/release/1

https://law.mit.edu/pub/deconstructinglegaltext/release/1

cohubicol's somewhat incoherent anxieties

[probably a few others]

\section{Theoretical Synthesis; Positioning}

Hypothesis: the "Minimum viable product" for expressing law needs to be actually pretty big.

\subsection{New research directions and desired features/requirements}

Turing-completeness vs decidability.

\section{Our Solution: L4}

In this section we demonstrate ``L4'',\footnote{No relation to \href{https://dl.acm.org/doi/10.1145/1629575.1629596}{seL4, the secure Linux kernel project}, apart from our common interest in formal verification.} a prototype domain-specific language (DSL) for drafting laws and contracts. It borrows ideas from the existing literature in languages and logics for rules and contracts, but is designed with productisation in mind. The concrete syntax of the language is intended to be user-friendly, and invites a new generation of legal engineers to attack the knowledge acquisition bottleneck with open-source methods. The interpreter supports interoperability with an ecosystem of third-party tools. In accordance with ``whole product'' theory, key augmentations include IDE support, intelligent feedback from formal verification engines, extraction to natural language representations, and transpilation to rule-engine runtimes.

\subsection{Case Study}

In this paper we visit two case studies: one a toy contract, and the other an extract from a legal regulation. Perhaps the Deon Bike example?

\subsection{High Level Design}

L4 draws its theoretical foundations from prior art in defeasible, default, and modal logics as applied to legal reasoning e.g. \cite{governatori_variants_2007}, and further builds on existing legal formalisms such as FormaLex\cite{gorin_software_2011}, Catala\cite{merigoux_catala_nodate}, DMN\cite{omg_decision_nodate}, Drools\cite{red_hat_drools_nodate}, LegalXML\cite{athan_legalruleml_2013}, Henglein's POETS CSL\cite{andersen_domain-specific_2014}, McCarty's LLD\cite{mccarty_language_1989}, and Schneider's $\mathcal{CL}$\cite{camilleri_contracts_2017}. Its primary contribution lies in synthesizing prior work into a DSL optimised for industry adoption. It is intended to be syntactically as user-friendly as SQL, and invites a new generation of legal engineers to apply open-source methods to the knowledge acquisition bottleneck. The interpreter supports inter-operability with an ecosystem of third-party tools. Key augmentations include IDE support, intelligent feedback from formal verification engines, extraction to natural language representations, and transpilation to operational rule engines.

L4's applied focus places it within the ``Rules as Code'' movement (e.g. OpenFisca\cite{openfisca_openfisca_nodate}, Catala\cite{merigoux_catala_nodate}) that itself draws on early computational law thinking (\cite{sergot_british_1986, love_computational_2005}). But rather than focusing on encoding laws into existing programming languages, we devise an external DSL designed for legal specification. This technology-demonstration paper illustrates L4 before enumerating its intended applications.\footnote{Code available at \url{https://github.com/smucclaw/complaw/doc/ex-jurix-20200814}.}

\subsection{Basic Illustration}

Suppose regulations decree, \textit{inter alia}, that cabbages may only be sold on days a full moon occurs unless the Director of Agriculture had granted an exemption. Buyers have the right, within three weeks of purchase, to return their cabbages for a 90\% refund, which the seller must issue within 3 days of the return. In L4, this might read:

\begin{lstlisting}
   IMPORT ContractLaw

   RULE  1 SaleRestricted
     PARTY NOBODY      AS P1       // variable binding to active party
       MAY sell Item   AS sale     // variable binding to action
      WHEN Item IS cabbage         // to be defined below
    UNLESS sale IS onLegalDate     // to be defined below
	OR UNLIKELY P1 HAS Exemption.from ~ [DirectorOfAgriculture]
     HENCE ReturnPolicy LEST VIOLATION   // next-state transitions
     WHERE Item IS cabbage     // "where" syntax borrowed from Haskell
	     WHEN Item.species ~ ["Brassica chinensis"|"Brassica oleracea"]
	    sale IS onLegalDate
	     WHEN sale.date ~ LegalDates
	     WHERE LegalDates = external(url="https://www.almanac.com/astronomy/moon/full/")

   RULE  2 ReturnPolicy
     GIVEN sale
     PARTY Buyer
       MAY return Item
    BEFORE sale.date + 3 weeks
     HENCE Net3

   RULE  3 Net3
     GIVEN return
     PARTY Seller
      MUST refund Amount
    BEFORE return.date + 3 days
     WHERE Amount = $return.sale.cash * 90%
\end{lstlisting}

\noindent With the rules thus encoded, L4's toolchain is now ready to generate English, extract interactive interviews that ask and answer questions from users, and formally verify and validate the rules.

\subsection{Key Toolchain Features}

\subsubsection{Natural Language Generation (NLG)}

Could Inari please revise this section to outline the individual application grammar components, and show how they fit together to construct a single long legal sentence; perhaps a sentence from Section 34 or from our Deon Bike example? Please discuss with Meng to choose the sentence.

Natural language contracts are needed for signature and for laying before a judge, while multi-lingual contracts are increasingly required by international commerce. Recalling Landin's \cite{landin_next_1966} remark that ``any pidgin algebra can be dressed up as pidgin English to please the generals'', Haskell might be considered the $\lambda$-calculus dressed as ASCII. Similarly, dressing up the $\mu$- and $\pi$-calculi produces L4.\footnote{With support for the $\lambda$, $\mu$, and $\pi$ calculi, L4 has enough going for it to call itself a LAMP stack for law.} To please Landin, L4 can be further dressed in natural language using Grammatical Framework \cite{ranta_grammatical_2004}.

This approach, which extends research in controlled natural languages \cite{fuchs_attempto_1996, angelov_implementing_2009}, is L4's way of achieving the \textit{isomorphism} goal \cite{bench-capon_isomorphism_2009} of computational law. For this demo, we developed pidgin grammars for English and Malay with the goal of producing fluent natural language text:

\begin{lstlisting}
$ l4 nlg cabbage.l4 --to=en_GB --to=ms_MY

No person shall sell a cabbage (an item with species Brassica chinensis
or Brassica oleracea) except on the day of a full moon, unless the seller
has an exemption granted by the Director of Agriculture. A buyer may
return their purchase within three weeks for a refund. The refund amount
is 90% of the sale price. The seller must issue the refund within 3 days.

Tidak ada orang yang boleh menjual kubis (sayu dari spesies Brassica
chinensis ora Brassica oleracea) kecuali pada hari bulan purnama, kecuali
pengecualian diberikan oleh Pengarah Pertanian. Pembeli mempunyai hak
untuk mengembalikan pembelian mereka dalam masa tiga minggu dari
pembelian dengan bayaran balik 90%. Penjual mesti mengeluarkan bayaran
balik dalam masa 3 hari.
\end{lstlisting}
https://www.overleaf.com/project/5f3225382862e30001114359
\noindent An IDE with realtime NLG would help L4 coders check their intuition against a natural language preview. Some believe that text produced in this fashion could even find its way into bills and contracts.

\subsubsection{Extraction to Rule Engines and Expert Systems}

Could Jason introduce DocAssemble as a modern expert system; and we show how we perform extraction to it, as a mix of Python and YAML.

Users frequently go to lawyers with questions like ``under what scenarios will I have a certain obligation?'' and ``a certain event has occurred; what do I have to do?'' Answering such questions could be the job of a user interface to a rule engine. Computable contracts \cite{surden_computable_2012} afford new ways of understanding one's obligations and rights: interactive dialogues and visualizations could substitute for asking a lawyer ``what if''.

To support such rule engines, L4 transpiles to Prolog, Typescript, and Python. As an example, L4's toolchain generates a mix of Python and YAML, to be read by Docassemble \cite{pyle_docassemble_nodate}, a modern web expert system. The web interface can be demonstrated by running a Docker instance of Docassemble and loading \texttt{cabbage.yaml} in its playground.

\subsubsection{Formal Verification: Cassandras in Code}

Could Martin introduce SAT solvers, model checkers, and interactive theorem provers briefly, and describe in a few sentences how they inspire us to offer useful features to developers in our language.

The first rule restricts a ``sale'', which one might define as a pattern of events in which two parties exchange what contract law calls ``consideration'': Party $P_A$ promises to transfer Item $I_A$ to Party $P_B$ in return for a promise that Party $P_B$ will transfer Item $I_B$ to Party $P_A$. (In this example, definitions of ``sale'' and ``consideration'' reside in a shared library \texttt{ContractLaw}, not detailed here for space reasons.)

Formal verification tools (e.g.\ \cite{larsen_uppaal_1997, sun_pat_2009}) excel at detecting such patterns. In the concept code below, L4 calls out to an external model checker which automatically detects a conflict between the rules given: surprisingly, a return meets the definition of a sale.

\begin{lstlisting}
$ l4 fv -Wall cabbage.l4
  Event trace -- suppose this sequence of events occurs:
    - 2020-01-10 Sale of cabbage from Alice to Bob for $10.
    - 2020-01-12 Return of cabbage from Bob to Alice.
    - 2020-01-13 Refund of $9 from Alice to Bob.
  Analysis:
    - Rule 2 determines that 2020-01-12 is within 3 weeks of 2020-01-10.
    - Rule 2 permits return of cabbage within 3 weeks.
    - On 2020-01-12 consideration (cabbage) moves from Buyer to Seller.
    - On 2020-01-13 consideration ($9) moves from Seller to Buyer.
    - Therefore a sale of cabbage occurs on 2020-01-12.
    - Rule 1 determines that on 2020-01-12 the moon is not full.
    - Rule 1 prohibits sale of cabbage on 2020-01-12.
    - The trace VIOLATES rule 1.
  General problem:
    - Every return counts as a sale.
\end{lstlisting}

\noindent In this admittedly contrived example, L4 exposes a potential drafting loophole: should returns constitute sales? Common sense would of course suggest that they should not. But legal disputes are no strangers to obtuse readings, and human drafters may not always have the presence of mind to look beneath obvious commercial readings to see loopholes exposed by obtuse readings. Thus computers, which precisely lack common sense, are well-placed to verify these rules. Drafters may refine the notion of ``sale'' in the upstream library \texttt{ContractLaw}, define \textit{lex specialis} for automatic prioritization, or, if the return--sale conflict is a feature and not a bug, leave things as is.

\subsection{Future directions}

Developmental features of L4 under construction include: inter-operability with DMN decision models, BPMN process models, and LegalRuleML\cite{athan_legalruleml_2013}; a novel succinct syntax for disjunctive and conjunctive lists; a relational syntax for record attributes; and support for locale-specific legal jurisdiction libraries and dialect specialization of natural language output. The type inference component and underlying logics of the language will be detailed in a future paper. It is anticipated that ``lossless'' embedding of contract parameters and semantics into generated PDFs via the Extensible Metadata Protocol (XMP) will benefit the contract drafting, analytics, and lifecycle management industries.

\section{Conclusion}



% \noindent Here's an example citation\footcite{ries:sweeney} using the \verb|\footcite{...}| command from the \texttt{biblatex} package. There isn't a good, up-to-date BibTeX style for the Chicago style, so we're using the \texttt{biblatex} style \texttt{biblatex-chicago} instead. This means you'll need to run \texttt{biber} instead of \texttt{bibtex} if you're compiling this template on your local \LaTeX{} installation: on Overleaf, \texttt{biber} is run automatically. Note also that INO uses the Chicago author-year referencing style, but uses footnote citations. You can add pre-notes\footcite[See also][]{risse:2000} and post-notes\footcite[52]{checkel:1997}\footcite[119]{grieco:1993} with citations\footcite{simmons:2001} too, as well as multiple citations\footcite{elliot:1994,kilroy:1995,pauwelyn:2015,mastanduno:1996} in a single \verb|\footcite{...}|. Sometimes you may need to use\footnote{For an alternate treatment see \cite{ussenate:chemwarfare:1984} instead.} \verb|\cite{...}| in a \verb|\footnote{...}| as well. 

% For a source with more than three authors,\footcite{inman:etal:2013} the note should include only the first author's name and et al. When more than one work by the same author is cited, that author's name is written out for each bibliographic entry, not replaced with ---.

% No entry in the reference list is needed for newspaper or magazine articles; instead, include relevant information in a footnote.\footnote{\emph{Los Angeles Times}, 3 May 1993, A1} Article titles and authors are omitted except when including them would enhance understanding of points made in the text or the source.

% Similarly, no entry in the reference list is needed for unpublished interviews. Instead, include relevant information in a footnote.\footnote{Author's interview with James Murphy, Washington, DC, July 1990} If the interviewee was promised anonymity, describe Guidelines for Contributors 253 the informant as precisely as possible---for example, as a member of a category of individuals---without identifying the person.

% \paragraph{Author Anonymization}
% Omit self-references that reveal your identity. You can modestly cite your own work without using pronouns that reveal your identity. Do not use references to ``Author'' when these are likely to reveal your identity. If needed, you can use the \texttt{censor} package\footnote{\url{https://ctan.org/pkg/censor}} to redact portions of text that may identify the authors.

% \section{Example of a first section}

% Figure \ref{fig:example} shows a normal figure, while figure \ref{fig:twosubs} show one made up of two sub-figures. Figure \ref{fig:landscape} is an example of a landscaped figure. You can use\footcite{simmons:2014} the \verb|\floatnotes{...}| command to add notes below figures\footcite{halimi:1990}.

% \begin{table}
% \caption{Automobile Land Speed Records (GR 5-10).}\label{tab:example}

% \begin{tabular}{@{} l l l l r @{}}
% \toprule
% % \headrow
% \emph{Speed (mph)} & \emph{Driver} & \emph{Car} & \emph{Engine} & \emph{Date}     \\
% \midrule[\heavyrulewidth] %% Required by INO
% 407.447     & Craig Breedlove & Spirit of America          & GE J47    & 8/5/63   \\
% 413.199     & Tom Green       & Wingfoot Express           & WE J46    & 10/2/64  \\
% 434.22      & Art Arfons      & Green Monster              & GE J79    & 10/5/64  \\
% 468.719     & Craig Breedlove & Spirit of America          & GE J79    & 10/13/64 \\
% 526.277     & Craig Breedlove & Spirit of America          & GE J79    & 10/15/65 \\
% 536.712     & Art Arfons      & Green Monster              & GE J79    & 10/27/65 \\
% 555.127     & Craig Breedlove & Spirit of America, Sonic 1 & GE J79    & 11/2/65  \\
% 576.553     & Art Arfons      & Green Monster              & GE J79    & 11/7/65  \\
% 600.601     & Craig Breedlove & Spirit of America, Sonic 1 & GE J79    & 11/15/65 \\
% 622.407     & Gary Gabelich   & Blue Flame                 & Rocket    & 10/23/70 \\
% 633.468     & Richard Noble   & Thrust 2                   & RR RG 146 & 10/4/83  \\
% 763.035     & Andy Green      & Thrust SSC                 & RR Spey   & 10/15/97\\
% \bottomrule
% \end{tabular}

% \floatnotes{Data from \url{https://www.sedl.org/afterschool/toolkits/science/pdf/ast_sci_data_tables_sample.pdf}}

% \end{table}

% Lorem ipsum dolor sit amet, consectetur adipiscing elit, sed do eiusmod tempor incididunt ut labore et dolore magna aliqua. Ut enim ad minim veniam, quis nostrud exercitation ullamco laboris nisi ut aliquip ex ea commodo consequat. Duis aute irure dolor in reprehenderit in voluptate velit esse cillum dolore eu fugiat nulla pariatur. Excepteur sint occaecat cupidatat non proident, sunt in culpa qui officia deserunt mollit anim id est laborum.

% Lorem ipsum dolor sit amet, consectetur adipiscing elit, sed do eiusmod tempor\footnote{Example of a normal footnote} incididunt ut labore et dolore magna aliqua. Ut enim ad minim veniam, quis nostrud exercitation ullamco laboris nisi ut aliquip ex ea commodo consequat. Duis aute irure dolor in reprehenderit in voluptate velit esse cillum dolore eu fugiat nulla pariatur. Excepteur sint occaecat cupidatat non proident, sunt in culpa qui officia deserunt mollit anim id est laborum.

% Lorem ipsum dolor sit amet, consectetur adipiscing elit, sed do eiusmod tempor inciddt enim ad minim veniam, quis nostrud exercitation ullamco laboris nisi ut aliquip ex ea commodo consequat. 


% \begin{figure}
% \centering
% \includegraphics[width=0.8\textwidth]{example-image}
% \floatnotes{Here are some notes. Here are some notes. Here are some notes. Here are some notes. Here are some notes. Here are some notes. Here are some notes. Here are some notes. Here are some notes. Here are some notes.}
% \caption{This is a figure caption}
% \label{fig:example}
% \end{figure}


% \begin{figure}[bt!]
% \begin{minipage}{0.47\textwidth}
% \includegraphics[width=\linewidth]{example-image}
% \floatnotes[]{A brief note for a subfigure.\\\emph{Source:} the source}
% \end{minipage}
% \hfill
% \begin{minipage}{0.47\textwidth}
% \includegraphics[width=\linewidth]{example-image}
% \floatnotes[]{A brief note for a subfigure.\\\emph{Source:} the source}
% \end{minipage}

% \caption{This is a caption for the entire figure}
% \label{fig:twosubs}
% \end{figure}

% \begin{sidewaysfigure}
% \centering
% \includegraphics[width=14cm]{example-image}
% \floatnotes{Here are some notes. Here are some notes. Here are some notes. Here are some notes. Here are some notes. Here are some notes. Here are some notes. Here are some notes. Here are some notes. Here are some notes.}
% \caption{This is a figure caption}
% \label{fig:landscape}
% \end{sidewaysfigure}

% \blinddocument

% A supplementary material section will always appear before the (optional) Appendix and Reference list. If you're certain that your submission won't have any supplementary material, you can add the \texttt{nosupp} option to the document class declaration, i.e. 
% \begin{quote}
% \verb|\documentclass[nosupp]{cup-ino}|    
% \end{quote}

% %% Optional appendix
% \appendix

% \begin{figure}[hbt!]
%     \centering
%     \includegraphics[width=4cm]{example-image}
%     \caption{Appendix figure}
%     \label{app:fig}
% \end{figure}

% \begin{table}[hbt!]
%     \caption{Appendix table}
%     \label{app:tab1}
% \begin{tabular}{c c c c c}
% \toprule
%      abc & def & ghi & jkl & mno \\
% \midrule
%      abc & def & ghi & jkl & mno \\
%      abc & def & ghi & jkl & mno \\
%      abc & def & ghi & jkl & mno \\
%      abc & def & ghi & jkl & mno \\
%      abc & def & ghi & jkl & mno \\
% \bottomrule
% \end{tabular}
% \end{table}


\printbibliography

\section*{Acknowledgements}
We thank\ldots

\section*{Key Words}
Keyword one; keyword two; keyword three; keyword four.

\end{document}
